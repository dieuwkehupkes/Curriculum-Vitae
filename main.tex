%% start of file `template.tex'.
%% Copyright 2006-2013 Xavier Danaux (xdanaux@gmail.com).
%
% This work may be distributed and/or modified under the
% conditions of the LaTeX Project Public License version 1.3c,
% available at http://www.latex-project.org/lppl/.

\documentclass[11pt,a4paper,roman, colorlinks, linkcolor=true]{moderncv}        % possible options include font size ('10pt', '11pt' and '12pt'), paper size ('a4paper', 'letterpaper', 'a5paper', 'legalpaper', 'executivepaper' and 'landscape') and font family ('sans' and 'roman')

% modern themes
\moderncvstyle{banking}                            % style options are 'casual' (default), 'classic', 'oldstyle' and 'banking'
\moderncvcolor{blue}                                % color options 'blue' (default), 'orange', 'green', 'red', 'purple', 'grey' and 'black'
%\renewcommand{\familydefault}{\sfdefault}         % to set the default font; use '\sfdefault' for the default sans serif font, '\rmdefault' for the default roman one, or any tex font name
\nopagenumbers{}                                  % uncomment to suppress automatic page numbering for CVs longer than one page

% character encoding
\usepackage[utf8]{inputenc}
\usepackage{fontawesome}
\usepackage{tabularx}
\usepackage{ragged2e}
\usepackage{comment}
% if you are not using xelatex ou lualatex, replace by the encoding you are using
%\usepackage{CJKutf8}                              % if you need to use CJK to typeset your resume in Chinese, Japanese or Korean

% adjust the page margins
\usepackage[scale=0.8]{geometry}
\usepackage{multicol}

% \usepackage{hyperref} % Required for adding links	and customizing them
\definecolor{linkcolour}{rgb}{0,0.2,0.6} % Link color
%\setlength{\hintscolumnwidth}{3cm}                % if you want to change the width of the column with the dates
%\setlength{\makecvtitlenamewidth}{10cm}           % for the 'classic' style, if you want to force the width allocated to your name and avoid line breaks. be careful though, the length is normally calculated to avoid any overlap with your personal info; use this at your own typographical risks...

\usepackage{import}

% personal data
\name{Dieuwke}{Hupkes}
%\title{Curriculum Vitae}                               % optional, remove / comment the line if not wanted
% \address{Delistraat 24H, Amsterdam}{}{}% optional, remove / comment the line if not wanted; the "postcode city" and and "country" arguments can be omitted or provided empty
% \phone[mobile]{+31 613679766}                   % optional, remove / comment the line if not wanted
% \phone[fixed]{01234 123456}                    % optional, remove / comment the line if not wanted
%\phone[fax]{+3~(456)~789~012}                      % optional, remove / comment the line if not wanted
% \email{xpan1@swarthmore.edu}                               % optional, remove / comment the line if not wanted
% \homepage{shawnpan.me}                         % optional, remove / comment the line if not wanted
% \extrainfo{}                 % optional, remove / comment the line if not wanted
%\photo[64pt][0.4pt]{picture}                       % optional, remove / comment the line if not wanted; '64pt' is the height the picture must be resized to, 0.4pt is the thickness of the frame around it (put it to 0pt for no frame) and 'picture' is the name of the picture file
%\quote{Some quote}                                 % optional, remove / comment the line if not wanted

% to show numerical labels in the bibliography (default is to show no labels); only useful if you make citations in your resume
%\makeatletter
%\renewcommand*{\bibliographyitemlabel}{\@biblabel{\arabic{enumiv}}}
%\makeatother
%\renewcommand*{\bibliographyitemlabel}{[\arabic{enumiv}]}% CONSIDER REPLACING THE ABOVE BY THIS

% bibliography with mutiple entries
%\usepackage{multibib}
%\newcites{book,misc}{{Books},{Others}}
  
\newcommand*{\customcventry}[7][.25em]{
  \begin{tabular}{@{}l} 
    {\bfseries #4}
  \end{tabular}
  \hfill% move it to the right
  \begin{tabular}{l@{}}
     {\bfseries #5}
  \end{tabular} \\
  \begin{tabular}{@{}l} 
    {\itshape #3}
  \end{tabular}
  \hfill% move it to the right
  \begin{tabular}{l@{}}
     {\itshape #2}
  \end{tabular}
  \ifx&#6&%
  \else{\\%
    \begin{minipage}{0.75\maincolumnwidth}%
      \small#6%
    \end{minipage}}\fi%
  \par\addvspace{#1}\vspace{1.7mm}}

\newcommand*{\customcvproject}[4][.25em]{
%   \vfill\noindent
  \begin{tabular}{@{}l} 
    {\bfseries #2}
  \end{tabular}
  \hfill% move it to the right
  \begin{tabular}{l@{}}
     {\itshape #3}
  \end{tabular}
  \ifx&#4&%
  \else{\\%
    \textit{\begin{minipage}{0.75\maincolumnwidth}%
      \small#4%
    \end{minipage}}}\fi%
  \par\addvspace{#1}\vspace{0mm}}

\setlength{\tabcolsep}{12pt}

%----------------------------------------------------------------------------------
%            content
%----------------------------------------------------------------------------------
\begin{document}

% set link colour
\hypersetup{colorlinks,breaklinks,urlcolor=linkcolour,linkcolor=linkcolour} % Set link colors throughout the document

%\begin{CJK*}{UTF8}{gbsn}                          % to typeset your resume in Chinese using CJK
%-----       resume       ---------------------------------------------------------
% \hypersetup{colorlinks,breaklinks,urlcolor=linkcolour,linkcolor=linkcolour} % Set link colors throughout the document
\makecvtitle
\vspace*{-20mm}

\begin{center}
\begin{tabular}{ c c c c }
 \faGlobe\enspace dieuwkehupkes.nl & \faEnvelopeO\enspace d.hupkes@uva.nl & \faGithub\enspace dieuwkehupkes &  \faMobile\enspace +31 613679766\\  
\end{tabular}
\end{center}

\section{EMPLOYMENT}

{\customcventry{January 2019 - present}{Temporary lecturer}{ILLC, University of Amsterdam}{Amsterdam}{}{ %\begin{itemize}
    % \item Advisor: Dr. Willem Zuidema
    % \item[] {\footnotesize Topic: Hierarchical compositionality in neural networks}
% \end{itemize}
}}

{\customcventry{July 2015 - January 2019}{PhD Student in the \href{https://www.languageininteraction.nl}{Language in Interaction} Consortium}{ILLC, University of Amsterdam}{Amsterdam}{\begin{itemize}
    \item Advisor: Dr. Willem Zuidema
    \item[] {\footnotesize Topic: Interpretability and hierarchical compositionality in neural networks}
\end{itemize}}{}}

{\customcventry{January - April 2019}{Research Intern}{Facebook AI Research}{Paris}{\begin{itemize}
    \item Advisors: Diane Bouchacourt and Marco Baroni
    \item[] {\footnotesize Topic: Assessing compositionality of languages emerging in referential games}
\end{itemize}}{}}

{\customcventry{February 2014 - June 2015}{Research Assistant}{ILLC, University of Amsterdam}{Amsterdam}{\begin{itemize}
    \item Under supervision of Dr. Willem Zuidema
    \item[] {\footnotesize Topic: Neural models of parsing}
\end{itemize}}{}}

{\customcventry{February 2014 - June 2015}{Pre-PhD fellowship}{CREATE, University of Amsterdam}{Amsterdam}{\begin{itemize}
    \item Under supervision of Prof. Dr. Rens Bod
    \item[] {\footnotesize Topic: Part-of-Speech tagging of 17th century Dutch}
\end{itemize}}{}}


\section{EDUCATION}
{\customcventry{July 2015 - May 2020}{Doctorate degree}{University of Amsterdam}{Amsterdam}{}{\begin{itemize}
    \item Dissertation: Hierarchy and interpretability in neural modeles of language processing
    \item[] {\footnotesize Promotores: Willem Zuidema, Rens Bod}
\end{itemize}}}

{\customcventry{September 2011 - December 2013}{Master of Logic}{University of Amsterdam}{Amsterdam}{}{\begin{itemize}
    \item Thesis: An empirical account of compositionality of translation through translation data
    \item[] {\footnotesize Supervisor: Khalil Sima'an}
\end{itemize}}}

{\customcventry{Fall 2012}{Exchange Semester}{University of Edinburgh}{Edinburgh}{}{}
}

{\customcventry{Sept 2010 - June 2011}{Preparation year}{University of Amsterdam}{Amsterdam}{}{}
}

{\customcventry{2006 - 2010}{Bachelor of Science in Physics and Astronomy}{University of Amsterdam}{Amsterdam}{}{\begin{itemize}
    \item Dissertation: Bohr's Atomic model: the evolution of a theory
    \item[] {\footnotesize Supervisors: Dr. A.J. Kox, Dr. E.P. Verlinde}
\end{itemize}}}


\section{RESEARCH INTERESTS}

\begin{itemize}
    \item Computational and cognitive models of natural language processing
    \item Hierarchy and compositionality in artificial neural networks
    \item Learning biases
    \item Statistical parsing, syntax
    \item Neurocomputational models of language (processing)
    \item language emergence
\end{itemize}


\section{TEACHING EXPERIENCE}

\subsection{As thesis supervisor}
{\customcvproject{Jaap Jumelet -- Msc Artificial Intelligence}{Jan 2019 – now}
{%\begin{itemize}
  %\item[] \textit{Recoding latent sentence representations}
  %\item[] Co-supervision with Willem Zuidema %,second reader: Willem Zuidema
%\end{itemize}
}}

{\customcvproject{Gautier Dagan -- Msc Artificial Intelligence}{Jan – Aug 2019}
{\begin{itemize}
  \item[] \textit{Co-Evolution of Language and Agent in Referential Games}
  %\item[] Co-supervision with Elia Bruni, second reader: Willem Zuidema
\end{itemize}
}

{\customcvproject{Dennis Ulmer -- Msc Artificial Intelligence}{Jan – Aug 2019}
{\begin{itemize}
  \item[] \textit{Recoding latent sentence representations}
  %\item[] Co-supervision with Elia Bruni, second reader: Willem Zuidema
\end{itemize}
}

{\customcvproject{Diana Rodriguez Luna -- Msc Artificial Intelligence}{Jan – July 2019}
{\begin{itemize}
  \item[] \textit{Language emergence in multi-agent referential games}
  %\item[] Co-supervision with Elia Bruni, second reader: Piek Vossen
\end{itemize}
}

{\customcvproject{Kris Korrel -- Msc Artificial Intelligence}{Jan – Nov 2018}
{\begin{itemize}
  \item[] \textit{From sequence to attention}
  %\item[] Co-supervision with Elia Bruni, second reader: Efstratios Gavves
\end{itemize}
}

{\customcvproject{Sanne Bouwmeester -- Msc Artificial Intelligence}{Jan – Oct 2018}
{\begin{itemize}
  \item[] \textit{Analysing seq-to-seq models in goal oriented dialogue: generalising to disfluencies}
  %\item[] Co-supervision with Raquel Fernandez, second reader: Ekaterina Shutova
\end{itemize}
}

{\customcvproject{Krstó Proroković -- Master of Logic}{Jan – Nov 2018}
{\begin{itemize}
  \item[] \textit{Learning to decide a formal language: a recurrent neural network approach}
  %\item[] Co-supervision with Elia Bruni and Germán Kruszewski
\end{itemize}
}

{\customcvproject{Anand Kumar Singh -- Msc Artificial Intelligence}{Jan  – Aug 2018}
{\begin{itemize}
  \item[] \textit{Pondering in artificial neural networks}
  %\item[] Co-supervision with Elia Bruni, second reader: Willem Zuidema
\end{itemize}}}

{\customcvproject{Ujjwal Sharma -- Msc Artificial Intelligence}{Jan – Aug 2018}
{\begin{itemize}
  \item[] \textit{Interpreting decision-making in interactive visual dialogue}
  %\item[] Co-supervision with Elia Bruni, second reader: Raquel Fernández
\end{itemize}}}

{\customcvproject{Rezka Aufar Leonandya -- Msc Artificial Intelligence}{Jan – Aug 2018}
{\begin{itemize}
  \item[] \textit{Learning to follow instructions}
  %\item[] Co-supervision with Elia Bruni and Germán Kruszewski, second reader: Willem Zuidema
\end{itemize}}}

{\customcvproject{Lucas Weber -- Msc Brain and Cognitive Science}{Jan – Jun 2018}
{\begin{itemize}
  \item[] \textit{Continual learning in humans and neuroscience-inspired AI}
  %\item[] Co-supervision with Elia Bruni and Germán Kruszewski, second reader: Willem Zuidema
\end{itemize}}}

{\customcvproject{Philip Bouman -- Bsc Artificial Intelligence}{Jan – Aug 2018}
{\begin{itemize}
  \item[] \textit{Modelling fonts with convolutional neural networks}
  %\item[] Co-supervision with Willem Zuidema
\end{itemize}}}

\vspace{1mm}

\subsection{As supervisor of individual/group projects}
{\customcvproject{Syntactic Awareness in Language Models: Recurrence vs Self-Attention}{Jun 2019}
{\begin{itemize}
  %\item Joint supervision with Elia Bruni
  \item[] \textit{Msc AI students Sander Bos, Lorian Colthof, Bryan Guevara and Vivian van Oijen}
  %\item[] Co-supervision with Willem Zuidema
\end{itemize}}}

{\customcvproject{Unsupervised Grammar Induction in Emergent Languages}{Jun 2019}
{\begin{itemize}
  %\item Joint supervision with Elia Bruni
  \item[] \textit{Msc AI students Silvan de Boer and Oskar van der Wal}
  %\item[] Co-supervision with Willem Zuidema
\end{itemize}}}

{\customcvproject{On the Realisation of Compositionality in Neural Networks}{Jun 2018}
{\begin{itemize}
  %\item Joint supervision with Elia Bruni
  \item[] \textit{Msc AI students Joris Baan, Jana Leible, Mitja Nikolaus, David Rau, Verna Dankers, Santhosh Rajamanickam and Dennis Ulmer}
  %\item[] Co-supervision with Willem Zuidema
\end{itemize}}}

{\customcvproject{Analysing Subject-Verb agreement with Diagnostic Classification}{Jun 2018}
{\begin{itemize}
  %\item Joint supervision with Elia Bruni
  \item[] \textit{Msc AI students Mario Giulianelli, Jack Harding and Florian Mohnert}
  %\item[] Co-supervision with Willem Zuidema
\end{itemize}}}

{\customcvproject{What do language models encode?}{Jun 2018}
{\begin{itemize}
  %\item Joint supervision with Elia Bruni
  \item[] \textit{Msc AI student Jaap Jumelet}
  %\item[] Co-supervision with Willem Zuidema
\end{itemize}}}


{\customcvproject{Learning compositionality in Neural Networks}{Jan 2018}
{\begin{itemize}
  %\item Joint supervision with Elia Bruni
  \item[] \textit{Master of Logic students Federico Schiaffino, Haukur Pál Jónsson, Max Rapp, Flavio Tisi and Yuan-Ho Yao}
  %\item[] Co-supervision with Willem Zuidema
\end{itemize}}}



\subsection{Guest lectures}

\begin{itemize}
\setlength\itemsep{3pt}
    \item \textbf{Statistical Methods for Natural Language Semantics} \hfill \textit{May 2019}
    \item \textbf{Foundations of Neural and Cognitive Modelling} \hfill \textit{November 2018}
    \item \textbf{Cognitive Models of Language and beyond} \hfill \textit{March 2018}
    \item \textbf{Natural Language Processing 1} \hfill \textit{Dec 2017}
    \item \textbf{Cognitive Models of Language and Music} \hfill \textit{Mar 2017}
\end{itemize}

\vspace{2mm}

\begin{comment}
{\customcvproject{Semantics}{May 2019}
{%\begin{itemize}\item[] \textit{Interpretability in Deep Learning} \end{itemize}
}}
{\customcvproject{Foundations of Neural and Cognitive Modelling}{November 2018}{}}
{\customcvproject{Natural Language Processing 1}{Dec 2017}{}}
{\customcvproject{Cognitive Models of Language and Music}{Mar 2017}{}}
\end{comment}

\subsection{As teaching assistant}

{\customcvproject{Natural Language Processing 1}{Oct-Dec 2017}
{Msc Artificial Intelligence, Master of Logic}
{\customcvproject{Gomputational Semantics and Pragmatics}{Sept-Oct 2016}
{Msc Artificial Intelligence, Master of Logic}
{\customcvproject{Evolution of Language and Music}{Feb-Apr 2016, Oct-Dec 2016}
{Bsc Psychobiologie}
{\customcvproject{Foundations of Neural and Cognitive Modelling}{Oct-Dec 2015}
{Msc Artificial Intelligence, Master of Logic, Msc Brain and Cognitive Science}
{\customcvproject{Unsupervised Language Learning}{Feb-Apr 2014, Feb-Apr 2015}
{Msc Artificial Intelligence, Master of Logic}
{\customcvproject{Automata and Formal Languages}{Apr-Jun 2013, Apr-Jun 2012}
{Bsc Artificial Intelligence}
{\customcvproject{Automata and Formal Languages}{Feb 2014, Feb 2013, Feb 2012}
{Bsc Beta-Gamma}
{\customcvproject{Biomechanics}{Feb-Jun 2013}
{Bsc bewegingswetenschappen (human motion sciences)}


\section{AWARDS AND FELLOWSHIPS}

{\customcvproject{Honourary mention}{CoNLL, 2023}
{The validity of evaluation results: assessing concurrence across compositionality benchmarks}

{\customcvproject{Honourary mention}{CoNLL, 2023}
{Mind the instructions; a holistic evaluation of consistency and interactions in prompt-based learning}

{\customcvproject{Best paper award}{MLRC, 2022}
{A replication study of compositional generalization works on semantic parsing}

{\customcvproject{Best paper award}{ConLL, 2021}
{Generalising to German prular noun classes, from the perspective of a recurrent neural network}

{\customcvproject{Honourary mention}{CoNLL, 2019}
{Analysing neural language models: contextual decomposition reveals default reasoning in number and gender assignment}

{\customcvproject{Best paper award}{BlackBoxNLP, 2018}
{Under the hood: using diagnostic classifiers to investigate and improve how language models track agreement information}

{\customcvproject{Research Internship}{2019}
{With Marco Baroni, at Facebook AI Research}


{\customcvproject{Scholarship for Doctorate Studies}{2015}
{With Willem Zuidema, in the Language in Interaction Consortium}

{\customcvproject{Pre-PhD fellowship}{2015}
{With Rens Bod, within CREATE}


\section{SERVICES}
\subsection{Organisation}

{\customcventry{November 2023}{\href{https://genbench.org/workshop}{EMNLP workshop on (benchmarking) generalisation in NLP}}{GenBench workshop 2023}{Singapore}
{\textbf{Role}: Lead organisor}{}}

{\customcventry{November 2022}{\href{https://blackboxnlp.github.io}{EMNLP workshop on analysing and interpreting neural networks}}{BlackboxNLP 2022 - Analyzing an interpreting Neural Networks}{Abu Dabhi}
{\textbf{Role}: Lead organisor}{}}

{\customcventry{November 2021}{\href{https://blackboxnlp.github.io}{EMNLP workshop on analysing and interpreting neural networks}}{BlackboxNLP 2021 - Analyzing an interpreting Neural Networks}{Hybrid}
{\textbf{Role}: Co-organiser}{}}

{\customcventry{November 2020}{\href{https://blackboxnlp.github.io}{EMNLP workshop on analysing and interpreting neural networks}}{BlackboxNLP 2020 - Analyzing an interpreting Neural Networks}{Virtual}
{\textbf{Role}: Co-organiser}{}}

{\customcventry{August 2019}{\href{https://www.lorentzcenter.nl/lc/web/2019/1120/info.php3?wsid=1120}{Workshop at the Lorentz workshop on compositionality}}{Compositionality in Brains and Machines}{Leiden}
{\textbf{Role}: Lead organisor with Willem Zuidema and Marco Baroni.}{}}

{\customcventry{August 2019}{\href{https://blackboxnlp.github.io/2019/}{ACL workshop on analysing and interpreting neural networks}}{BlackboxNLP 2019 - Analyzing an interpreting Neural Networks}{Florence}{\textbf{Role}: Co-organisation with Yonatan Belinkov, Grzegorz Chrupala and Tal Linzen.}{}}

{\customcventry{December 2017}{\href{https://smartcs.uva.nl/content/events/workshops/smart-conference-2017/12/workshop-4.html}{SMART workshop in honour of the scientific legacy of Remko Scha}}{Grammar, Computation and Cognition}{Amsterdam}{\textbf{Role}: Co-organisation with Willem Zuidema.}{}}

\subsection{Reviewing}
\begin{itemize}\setlength\itemsep{1mm}
    \item \textbf{Conferences and workshops}: ICLR, Neurips, ACL, EACL, EMNLP, NAACL, CoNLL, BlackboxNLP, Gecko, Student Workshops
    \item \textbf{Journals}: Computational Linguistics, TACL, Nature Machine Intelligence, Congitive Systems Research
    \item \textbf{As (senior) Area Chair}: EACL, NAACL, EMNLP, ACL, ARR
\end{itemize}

% \subsection{Other}
% \begin{itemize}
%     \item PhD council ILLC, University of Amsterdam
%     \item PhD council Faculty of Science, University of Amsterdam
% \end{itemize}


\section{TALKS \& PANELS}

\begin{itemize}
    \setlength\itemsep{5pt}
    \item \textit{March 17, 2021}, Kunnen we kunstmatige intelligentie nog doorgronden? Studium Generale, \textit{Utrecht} (virtual talk)
    \item \textit{February 11, 2021}, Compositionality decomposed: how do neural networks generalise? Women@CL, \textit{University of Cambridge} (virtual talk)
    \item \textit{February 5, 2021}, Compositionality decomposed: how about natural data? \textit{Rijksuniversiteit Groningen} (virtual talk)
        \item \textit{August}. TBA. Keynote speaker at the workshop Computational and experimental explanations in semantics and pragmatics, \textit{Utrecht}. \textcolor{red}{ -- Post-poned because of COVID19}
        \item \textit{November}. TBA. Keynote speaker at EurNLP, \textit{Paris}. \textcolor{red}{ -- Post-poned because of COVID19}
    \item \textit{October 31, 2020}, Neural networks as explanatory models of language processing, ILCC Seminar at the University of Edinburgh, \textit{Edinburgh} (virtual talk)
    \item \textit{September 17, 2020}, Neural networks as explanatory models, AllenNLP, \textit{Seattle} (virtual talk)
    \item \textit{November 15, 2019}. Syntax in neural language models: a case study \textit{University of Utrecht, Utrecht}
    \item \textit{October 9, 2019}. Subject verb agreement in neural language models -- how, when and where? \textit{Johns Hopkins University, Baltimore}
    \item \textit{October 1, 2019}. What do they learn? Neural networks, compositionality and interpretability. \textit{Computational Cognition workshop, Osnabruek}.
    \item \textit{September 3, 2019}. Guest speaker and panelist at the public event When fake looks all too real: the technology behind Deep Fake, \textit{SPUI25, Amsterdam}.
    \item \textit{August 1, 2019}. Blackbox NLP, \textbf{panel moderator}.
    \item \textit{June 18, 2019}. The typology of emergent languages. \textit{Interaction and the Evolution of Linguistic Complexity, Edinburgh}.
    \item \textit{May 6.} The compositionality of neural networks: integrating symbolism and connectionism. \textit{CS\&AI / SIKS workshop on analyzing and interpreting neural networks for NLP, ‘s-Hertogenbosch}.
    \item \textit{April 18}. The compositionality of neural networks: integrating symbolism and connectionism. \textit{Invited internal talk at Saarland University, Saarbrücken}.
    \item \textit{March 14}. On neural networks and compositionality. \textit{Invited internal seminar at École normale supérieure, Paris}.
    \item \textit{July 18, 2018}. Visualisation and ‘diagnostic classifiers’ reveal how recurrent and recursive neural networks process hierarchical structure. \textit{IJCAI, Stockholm}.
    \item \textit{June 12, 2018}. Learning compositionally through attentive guidance. \textit{Invited internal seminar at the University of Copenhagen.}
    \item \textit{May 9, 2017}. Processing hierarchical structure with RNNs.\textit{ Dagstuhl seminar on Human-like neural-symbolic computing.}
    \item \textit{December 7, 2017}. The grammar of neural networks. \textit{SMART workshop Grammars, Computation \& Cognition, Amsterdam.}
    \item \textit{December 15, 2017}. Hierarchical compositionality in recurrent neural networks. \textit{Invited internal seminar at Rijksuniversiteit Groningen.}
    \item \textit{May 25, 2016}. POS-tagging of Historical Dutch. \textit{LREC, Portoroz}.
    \item \textit{November 22, 2016}. How may neural networks process hierarchical structure? Insights from recursive and recurrent networks learning arithmetics. \textit{Logic Tea at the University of Amsterdam.}
    \item \textit{June 8, 2015} Using Parallel Data to improve Part-of-Speech tagging of 17th century Dutch. \textit{DH Benelux, Antwerp.}
\end{itemize}


\section{HIGHLIGHTED PUBLICATIONS}

A full overview of my publications can be found at my website (\textcolor{blue}{\url{https://dieuwkehupkes.nl/publications/}} or on my \textcolor{blue}{\href{https://scholar.google.com/citations?user=tAtSMTcAAAAJ&hl=en&oi=ao}{Google Scholar Page}}.\\

\begin{itemize}
\setlength\itemsep{5pt}
\item \textbf{D. Hupkes}, M. Giulianelli, V. Dankers, et. al (2023). \href{https://www.nature.com/articles/s42256-023-00729-y}{A taxonomy and review of generalization research in NLP}. Nature.
\item V. Dankers, E. Bruni, \textbf{D. Hupkes} \href{https://aclanthology.org/2022.acl-long.286/}{The paradox of the compositionality of natural language: a neural machine translation case study}. ACL.
\item V. Dankers, A. Langedijk, K. McCurdy, A. Williams, \textbf{D. Hupkes} (2021). \href{https://aclanthology.org/2021.conll-1.8/}{Generalising to German plural noun classes, from the perspective of a recurrent neural network}, CoNLL. \textbf{Best paper award}
\item \textbf{D. Hupkes}, V. Dankers, M. Mul, E. Bruni (2020). \href{https://www.jair.org/index.php/jair/article/view/11674/26576}{Compositionality decomposed:  how do neural networks generalise?}. JAIR.
\item \textbf{D. Hupkes}, S. Veldhoen, Zuidema, W. (2018). \href{https://jair.org/index.php/jair/article/view/11196/26408}{Visualisation and ‘diagnostic classifiers’ reveal how recurrent and recursive neural networks process hierarchical structure}. JAIR.
    \item Giulianelli, M., Harding, J., Mohnert, F., Hupkes, D. and Zuidema, W. (2018). \href{https://aclweb.org/anthology/W18-5426}{Under the hood: using diagnostic classifiers to investigate and improve how language models track agreement information.} \textit{BlackboxNLP 2018, ACL}. \textbf{Best paper award}.
\end{itemize}


\section{LANGUAGES}

\begin{tabular}{rlcrl}
\textsc{Dutch:} & Mothertongue && \textsc{French:} & Basic Knowledge\\
\textsc{English:} & Fluent && \textsc{Russian:} & Basic Knowledge\\
\textsc{Italian:} & Good command && \textsc{German:} & Basic Knowledge\\
\textsc{Spanish} & Basic command && \textsc{Frysian:} & Understanding\\
&\\
\end{tabular}



\section{ET CETERA}

I'm part of a competitive pole dance duo, we participated twice in the national championships, placing 1st and 2nd. In 2018, we also participated in the World Championship and arrived at place 10 of the world ranking. \url{https://instagram.com/duo_polenotti}.

% \section{REFERENCES}
% 
\begin{tabular}{ll}
\emph{Dr. Willem Zuidema} & \emph{Dr Marco Baroni} \\
Institute for Logic, Language and Computation & ICREA, Facebook AI Research\\
Universiteit van Amsterdam & \\
Room F2.45 & \\
Building F & \\
Science Park 107 & \\
1098 XG Amsterdam & \\
\href{mailto:w.h.zuidema@uva.nl}{w.h.zuidema@uva.nl}& \href{mailto:mbaroni@fb.com}{mbaroni@fb.com }\\\\
% \emph{Dr. K. Sima'an} & \emph{Prof. Dr. R. Bod} \\
% Institute for Logic, Language and Computation & Institute for Logic, Language and Computation\\
% Universiteit van Amsterdam & Universiteit van Amsterdam\\
% Room F2.06 & Room F2.04\\
% Building F & Building F\\
% Science Park 107 & Science Park 107\\
%1098 XG Amsterdam & 1098 XG Amsterdam\\
%\href{mailto:k.simaan@uva.nl}{k.simaan@uva.nl}& \href{mailto:rens.bod@gmail.com}{rens.bod@gmail.com}
\end{tabular}

% Publications from a BibTeX file without multibib
%  for numerical labels: \renewcommand{\bibliographyitemlabel}{\@biblabel{\arabic{enumiv}}}% CONSIDER MERGING WITH PREAMBLE PART
%  to redefine the heading string ("Publications"): \renewcommand{\refname}{Articles}
% \nocite{*}
% \bibliographystyle{plain}
% \bibliography{publications}                        % 'publications' is the name of a BibTeX file

% Publications from a BibTeX file using the multibib package
%\section{Publications}
%\nocitebook{book1,book2}
%\bibliographystylebook{plain}
%\bibliographybook{publications}                   % 'publications' is the name of a BibTeX file
%\nocitemisc{misc1,misc2,misc3}
%\bibliographystylemisc{plain}
%\bibliographymisc{publications}                   % 'publications' is the name of a BibTeX file

%-----       letter       ---------------------------------------------------------

\end{document}
