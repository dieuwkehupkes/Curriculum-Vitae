%% start of file `template.tex'.
%% Copyright 2006-2013 Xavier Danaux (xdanaux@gmail.com).
%
% This work may be distributed and/or modified under the
% conditions of the LaTeX Project Public License version 1.3c,
% available at http://www.latex-project.org/lppl/.

\documentclass[11pt,a4paper,roman, colorlinks, linkcolor=true]{moderncv}        % possible options include font size ('10pt', '11pt' and '12pt'), paper size ('a4paper', 'letterpaper', 'a5paper', 'legalpaper', 'executivepaper' and 'landscape') and font family ('sans' and 'roman')

% modern themes
\moderncvstyle{banking}                            % style options are 'casual' (default), 'classic', 'oldstyle' and 'banking'
\moderncvcolor{blue}                                % color options 'blue' (default), 'orange', 'green', 'red', 'purple', 'grey' and 'black'
%\renewcommand{\familydefault}{\sfdefault}         % to set the default font; use '\sfdefault' for the default sans serif font, '\rmdefault' for the default roman one, or any tex font name
\nopagenumbers{}                                  % uncomment to suppress automatic page numbering for CVs longer than one page

% character encoding
\usepackage[utf8]{inputenc}
\usepackage{fontawesome}
\usepackage{tabularx}
\usepackage{ragged2e}
\usepackage{comment}
% if you are not using xelatex ou lualatex, replace by the encoding you are using
%\usepackage{CJKutf8}                              % if you need to use CJK to typeset your resume in Chinese, Japanese or Korean

% adjust the page margins
\usepackage[scale=0.8]{geometry}
\usepackage{multicol}

% \usepackage{hyperref} % Required for adding links	and customizing them
\definecolor{linkcolour}{rgb}{0,0.2,0.6} % Link color
%\setlength{\hintscolumnwidth}{3cm}                % if you want to change the width of the column with the dates
%\setlength{\makecvtitlenamewidth}{10cm}           % for the 'classic' style, if you want to force the width allocated to your name and avoid line breaks. be careful though, the length is normally calculated to avoid any overlap with your personal info; use this at your own typographical risks...

\usepackage{import}

% personal data
\name{Dieuwke}{Hupkes}
%\title{Curriculum Vitae}                               % optional, remove / comment the line if not wanted
\address{Delistraat 24H, Amsterdam}{}{}% optional, remove / comment the line if not wanted; the "postcode city" and and "country" arguments can be omitted or provided empty
% \phone[mobile]{+31 613679766}                   % optional, remove / comment the line if not wanted
% \phone[fixed]{01234 123456}                    % optional, remove / comment the line if not wanted
%\phone[fax]{+3~(456)~789~012}                      % optional, remove / comment the line if not wanted
% \email{xpan1@swarthmore.edu}                               % optional, remove / comment the line if not wanted
% \homepage{shawnpan.me}                         % optional, remove / comment the line if not wanted
% \extrainfo{}                 % optional, remove / comment the line if not wanted
%\photo[64pt][0.4pt]{picture}                       % optional, remove / comment the line if not wanted; '64pt' is the height the picture must be resized to, 0.4pt is the thickness of the frame around it (put it to 0pt for no frame) and 'picture' is the name of the picture file
%\quote{Some quote}                                 % optional, remove / comment the line if not wanted

% to show numerical labels in the bibliography (default is to show no labels); only useful if you make citations in your resume
%\makeatletter
%\renewcommand*{\bibliographyitemlabel}{\@biblabel{\arabic{enumiv}}}
%\makeatother
%\renewcommand*{\bibliographyitemlabel}{[\arabic{enumiv}]}% CONSIDER REPLACING THE ABOVE BY THIS

% bibliography with mutiple entries
%\usepackage{multibib}
%\newcites{book,misc}{{Books},{Others}}
  
\newcommand*{\customcventry}[7][.25em]{
  \begin{tabular}{@{}l} 
    {\bfseries #4}
  \end{tabular}
  \hfill% move it to the right
  \begin{tabular}{l@{}}
     {\bfseries #5}
  \end{tabular} \\
  \begin{tabular}{@{}l} 
    {\itshape #3}
  \end{tabular}
  \hfill% move it to the right
  \begin{tabular}{l@{}}
     {\itshape #2}
  \end{tabular}
  \ifx&#6&%
  \else{\\%
    \begin{minipage}{0.75\maincolumnwidth}%
      \small#6%
    \end{minipage}}\fi%
  \par\addvspace{#1}\vspace{1.7mm}}

\newcommand*{\customcvproject}[4][.25em]{
%   \vfill\noindent
  \begin{tabular}{@{}l} 
    {\bfseries #2}
  \end{tabular}
  \hfill% move it to the right
  \begin{tabular}{l@{}}
     {\itshape #3}
  \end{tabular}
  \ifx&#4&%
  \else{\\%
    \textit{\begin{minipage}{0.75\maincolumnwidth}%
      \small#4%
    \end{minipage}}}\fi%
  \par\addvspace{#1}\vspace{0mm}}

\setlength{\tabcolsep}{12pt}

%----------------------------------------------------------------------------------
%            content
%----------------------------------------------------------------------------------
\begin{document}

% set link colour
\hypersetup{colorlinks,breaklinks,urlcolor=linkcolour,linkcolor=linkcolour} % Set link colors throughout the document

%\begin{CJK*}{UTF8}{gbsn}                          % to typeset your resume in Chinese using CJK
%-----       resume       ---------------------------------------------------------
% \hypersetup{colorlinks,breaklinks,urlcolor=linkcolour,linkcolor=linkcolour} % Set link colors throughout the document
\makecvtitle
\vspace*{-20mm}

\begin{center}
\begin{tabular}{ c c c c }
 \faGlobe\enspace dieuwkehupkes.nl & \faEnvelopeO\enspace d.hupkes@uva.nl & \faGithub\enspace dieuwkehupkes &  \faMobile\enspace +31 613679766\\  
\end{tabular}
\end{center}

\section{EMPLOYMENT}

{\customcventry{Current}{PhD Student in the \href{https://www.languageininteraction.nl}{Language in Interaction} Consortium}{ILLC, University of Amsterdam}{Amsterdam}{}{\begin{itemize}
    \item Advisor: Dr. Willem Zuidema
    \item[] {\footnotesize Topic: Hierarchical compositionality in neural networks}
\end{itemize}}}

{\customcventry{January - April 2019}{Research Intern}{Facebook AI Research}{Paris}{}{\begin{itemize}
    \item Advisors: Diane Bouchacourt and Marco Baroni
    \item[] {\footnotesize Topic: Assessing compositionality of languages emerging in referential games}
\end{itemize}}}

{\customcventry{February 2014 - June 2015}{Research Assistant}{ILLC, University of Amsterdam}{Amsterdam}{}{\begin{itemize}
    \item Under supervision of Dr. Willem Zuidema
    \item[] {\footnotesize Topic: Neural models of parsing}
\end{itemize}}}

{\customcventry{February 2014 - June 2015}{Pre-PhD fellowship}{CREATE, University of Amsterdam}{Amsterdam}{}{\begin{itemize}
    \item Under supervision of Prof. Dr. Rens Bod
    \item[] {\footnotesize Topic: Part-of-Speech tagging of 17th century Dutch}
\end{itemize}}}

\section{EDUCATION}
{\customcventry{September 2011 - December 2013}{Master of Logic}{University of Amsterdam}{Amsterdam}{}{\begin{itemize}
    \item Dissertation: An empirical account of compositionality of translation through translation data
    \item[] {\footnotesize Supervisor: Khalil Sima'an}
\end{itemize}}}

{\customcventry{Fall 2012}{Exchange Semester}{University of Edinburgh}{Edinburgh}{}{}
}

{\customcventry{Sept 2010 - June 2011}{Preparation year}{University of Amsterdam}{Amsterdam}{}{}
}

{\customcventry{2006 - 2010}{Bachelor of Science in Physics and Astronomy}{University of Amsterdam}{Amsterdam}{}{\begin{itemize}
    \item Dissertation: Bohr's Atomic model: the evolution of a theory
    \item[] {\footnotesize Supervisors: Dr. A.J. Kox, Dr. E.P. Verlinde}
\end{itemize}}}


\section{RESEARCH INTERESTS}

\begin{itemize}
    \item Computational and cognitive models of natural language processing
    \item Hierarchy and compositionality in artificial neural networks
    \item Learning biases
    \item Statistical parsing, syntax
    \item Neurocomputational models of language (processing)
    \item language emergence
\end{itemize}


\section{TEACHING EXPERIENCE}

I obtained the Dutch Basic Teaching Qualification (UTQ/BKO) at the University of Amsterdam.\\

\subsection{As main lecturer/coordinator}
{\customcvproject{Taaltheorie en Taalverwerking}{March-June 2020}
{Bsc Artificial Intelligence}

{\customcvproject{Human(e) AI}{March-June 2020}
{Elective IIS course}

\subsection{As PhD (co)Advisor}
{\customcvproject{Lucas Weber}{October 2019 – now}
{\begin{itemize}
  % \item[] \textit{Co-supervision with Elia Bruni}
  \item[] \textcolor{blue}{\normalfont Co-supervision with Elia Bruni\vspace{1mm}}
\end{itemize}
}}

{\customcvproject{Lovish Madaan}{September 2023 – now}
{\begin{itemize}
  % \item[] \textit{Co-supervision with Elia Bruni}
  \item[] \textcolor{blue}{\normalfont Academic supervisor: Pontus Stenetorp\vspace{1mm}}
\end{itemize}
}}

\subsection{As intern or resident manager}

{\customcvproject{Yusuf Kocyigit}{2023-2024}
{\begin{itemize}
  \item[] \textit{Assessing the impact of evaluation data contamination in LLMs}
\end{itemize}
}}

{\customcvproject{Kaiser Sun}{2022-2023}
{\begin{itemize}
  \item[] \textit{Construct validity of compositional generalisation datasets}
\end{itemize}
}}

{\customcvproject{Verna Dankers}{2022-2023}
{\begin{itemize}
  \item[] \textit{Memorisation in NMT models}
\end{itemize}
}}

{\customcvproject{Itay Itzhak}{2021-2022}
{\begin{itemize}
  \item[] \textit{Localist representations in NMT models}
\end{itemize}
}}

{\customcvproject{Maartje ter Hoeve}{2021}
{\begin{itemize}
  \item[] \textit{Interactive NLP}
\end{itemize}
}}

\subsection{As thesis supervisor}

{\customcvproject{Hugh Mee Wong -- Msc Artificial Intelligence}{2021 – 2023}
{\begin{itemize}
  \item[] \textit{Assessing Language Model Consistency on Multiple-Choice Tasks}
\end{itemize}
}}

{\customcvproject{Tom Kersten -- Msc Artificial Intelligence}{2020 – 2021}
{\begin{itemize}
  \item[] \textit{The interpretability of neural language models}
\end{itemize}
}}

{\customcvproject{Sylke Goosen -- Msc Artificial Intelligence}{2020 – 2021}
{\begin{itemize}
  \item[] \textit{Investigating neural language models}
\end{itemize}
}}

{\customcvproject{Jeroen Taal -- Bsc Artificial Intelligence}{2020}
{\begin{itemize}
  \item[] \textit{Pruning of neural language models for Dutch}
\end{itemize}
}}

{\customcvproject{Hugh-Mee Wong -- Bsc Artificial Intelligence}{2020}
{\begin{itemize}
  \item[] \textit{What do neural language models learn about Dutch syntax?}
\end{itemize}
}}

{\customcvproject{Oskar van der Wal -- Msc Artificial Intelligence}{2019 – 2020}
{\begin{itemize}
  \item[] \textit{The grammar of emergent languages}
  %\item[] Co-supervision with Willem Zuidema %,second reader: Willem Zuidema
\end{itemize}
}}

{\customcvproject{Oscar Ligthart -- Msc Artificial Intelligence}{2019 – 2020}
{\begin{itemize}
  \item[] \textit{Consistency and structure in emergent languages}
  %\item[] Co-supervision with Willem Zuidema %,second reader: Willem Zuidema
\end{itemize}
}}


{\customcvproject{Jaap Jumelet -- Msc Artificial Intelligence}{2019 – 2020}
{\begin{itemize}
  \item[] \textit{The interpretability of neural language models}
  %\item[] Co-supervision with Willem Zuidema %,second reader: Willem Zuidema
\end{itemize}
}}

{\customcvproject{Gautier Dagan -- Msc Artificial Intelligence}{2019}
{\begin{itemize}
  \item[] \textit{Co-Evolution of Language and Agent in Referential Games}
  %\item[] Co-supervision with Elia Bruni, second reader: Willem Zuidema
\end{itemize}
}

{\customcvproject{Dennis Ulmer -- Msc Artificial Intelligence}{2019}
{\begin{itemize}
  \item[] \textit{Recoding latent sentence representations}
  %\item[] Co-supervision with Elia Bruni, second reader: Willem Zuidema
\end{itemize}
}

{\customcvproject{Diana Rodriguez Luna -- Msc Artificial Intelligence}{2019}
{\begin{itemize}
  \item[] Language emergence in multi-agent referential games
  \item[] \textcolor{blue}{\normalfont In collaboration with Facebook AI Research\vspace{1mm}}
  %\item[] Co-supervision with Elia Bruni, second reader: Piek Vossen
\end{itemize}
}

{\customcvproject{Kris Korrel -- Msc Artificial Intelligence}{2018}
{\begin{itemize}
  \item[] \textit{From sequence to attention}
  \item[] \textcolor{blue}{\normalfont In collaboration with Facebook AI Research\vspace{1mm}}
  %\item[] Co-supervision with Elia Bruni, second reader: Efstratios Gavves
\end{itemize}
}

{\customcvproject{Sanne Bouwmeester -- Msc Artificial Intelligence}{2018}
{\begin{itemize}
  \item[] \textit{Analysing seq-to-seq models in goal oriented dialogue: generalising to disfluencies}
  %\item[] Co-supervision with Raquel Fernandez, second reader: Ekaterina Shutova
\end{itemize}
}

{\customcvproject{Krstó Proroković -- Master of Logic}{2018}
{\begin{itemize}
  \item[] \textit{Learning to decide a formal language: a recurrent neural network approach}
  \item[] \textcolor{blue}{\normalfont In collaboration with Facebook AI Research\vspace{1mm}}
  %\item[] Co-supervision with Elia Bruni and Germán Kruszewski
\end{itemize}
}

{\customcvproject{Anand Kumar Singh -- Msc Artificial Intelligence}{2018}
{\begin{itemize}
  \item[] \textit{Pondering in artificial neural networks}
  \item[] \textcolor{blue}{\normalfont In collaboration with Facebook AI Research\vspace{1mm}}
  %\item[] Co-supervision with Elia Bruni, second reader: Willem Zuidema
\end{itemize}}}

{\customcvproject{Ujjwal Sharma -- Msc Artificial Intelligence}{2018}
{\begin{itemize}
  \item[] \textit{Interpreting decision-making in interactive visual dialogue}
  %\item[] Co-supervision with Elia Bruni, second reader: Raquel Fernández
\end{itemize}}}

{\customcvproject{Rezka Aufar Leonandya -- Msc Artificial Intelligence}{2018}
{\begin{itemize}
  \item[] \textit{Learning to follow instructions}
  \item[] \textcolor{blue}{\normalfont In collaboration with Facebook AI Research\vspace{1mm}}
  %\item[] Co-supervision with Elia Bruni and Germán Kruszewski, second reader: Willem Zuidema
\end{itemize}}}

{\customcvproject{Lucas Weber -- Msc Brain and Cognitive Science}{2018}
{\begin{itemize}
  \item[] \textit{Continual learning in humans and neuroscience-inspired AI}
  %\item[] Co-supervision with Elia Bruni and Germán Kruszewski, second reader: Willem Zuidema
\end{itemize}}}

{\customcvproject{Philip Bouman -- Bsc Artificial Intelligence}{2018}
{\begin{itemize}
  \item[] \textit{Modelling fonts with convolutional neural networks}
  %\item[] Co-supervision with Willem Zuidema
\end{itemize}}}

\vspace{1mm}

\subsection{As supervisor of individual or group projects}

{\customcvproject{Overgeneralisation in neural sequence to sequence models}{2020}
{\begin{itemize}
  \item[] \textit{Msc AI student Anna Langendijk}
  % \item Joint supervision with Willem Zuidema
\end{itemize}}}

{\customcvproject{Generalised contextual decomposition for transformer models}{2020}
{\begin{itemize}
  \item[] \textit{Msc AI student Tom Kersten}
  % \item Joint supervision with Willem Zuidema
\end{itemize}}}


{\customcvproject{XAI: A conceptual framework for interpretability methods}{2019 - 2020}
{\begin{itemize}
  \item[] \textit{Msc Brain and Cognitive Science student Lewis O'Sullivan}
  % \item Joint supervision with Willem Zuidema
\end{itemize}}}

{\customcvproject{The compositionality of neural networks}{2018 - 2019}
{\begin{itemize}
  %\item Joint supervision with Elia Bruni
  \item[] \textit{Msc AI students Verna Dankers and Mathijs Mul}
\end{itemize}}}

{\customcvproject{Syntactic Awareness in Language Models: Recurrence vs Self-Attention}{2019}
{\begin{itemize}
  %\item Joint supervision with Elia Bruni
  \item[] \textit{Msc AI students Sander Bos, Lorian Colthof, Bryan Guevara and Vivian van Oijen}
  %\item[] Co-supervision with Willem Zuidema
\end{itemize}}}

{\customcvproject{Unsupervised Grammar Induction in Emergent Languages}{2019}
{\begin{itemize}
  %\item Joint supervision with Elia Bruni
  \item[] \textit{Msc AI students Silvan de Boer and Oskar van der Wal}
  %\item[] Co-supervision with Willem Zuidema
\end{itemize}}}

{\customcvproject{On the Realisation of Compositionality in Neural Networks}{2018}
{\begin{itemize}
  %\item Joint supervision with Elia Bruni
  \item[] \textit{Msc AI students Joris Baan, Jana Leible, Mitja Nikolaus, David Rau, Verna Dankers, Santhosh Rajamanickam and Dennis Ulmer}
  %\item[] Co-supervision with Willem Zuidema
\end{itemize}}}

{\customcvproject{Analysing Subject-Verb agreement with Diagnostic Classification}{2018}
{\begin{itemize}
  %\item Joint supervision with Elia Bruni
  \item[] \textit{Msc AI students Mario Giulianelli, Jack Harding and Florian Mohnert}
  %\item[] Co-supervision with Willem Zuidema
\end{itemize}}}

{\customcvproject{What do language models encode?}{2018}
{\begin{itemize}
  %\item Joint supervision with Elia Bruni
  \item[] \textit{Msc AI student Jaap Jumelet}
  %\item[] Co-supervision with Willem Zuidema
\end{itemize}}}


{\customcvproject{Learning compositionality in Neural Networks}{2018}
{\begin{itemize}
  %\item Joint supervision with Elia Bruni
  \item[] \textit{Master of Logic students Federico Schiaffino, Haukur Pál Jónsson, Max Rapp, Flavio Tisi and Yuan-Ho Yao}
  %\item[] Co-supervision with Willem Zuidema
\end{itemize}}}

% \subsection{Guest lectures}
% 
% \begin{itemize}
% \setlength\itemsep{3pt}
%     \item \textbf{Natural Language Processing 2} \hfill \textit{April 2020}
%     \item \textbf{Statistical Methods for Natural Language Semantics} \hfill \textit{May 2019}
%     \item \textbf{Foundations of Neural and Cognitive Modelling} \hfill \textit{November 2018}
%     \item \textbf{Cognitive Models of Language and beyond} \hfill \textit{March 2018}
%     \item \textbf{Natural Language Processing 1} \hfill \textit{Dec 2017}
%     \item \textbf{Cognitive Models of Language and Music} \hfill \textit{Mar 2017}
% \end{itemize}
% 
% \vspace{2mm}
% 
% \subsection{As teaching assistant}
% 
% {\customcvproject{Natural Language Processing 1}{Oct-Dec 2017}
% {Msc Artificial Intelligence, Master of Logic}
% {\customcvproject{Gomputational Semantics and Pragmatics}{Sept-Oct 2016}
% {Msc Artificial Intelligence, Master of Logic}
% {\customcvproject{Evolution of Language and Music}{Feb-Apr 2016, Oct-Dec 2016}
% {Bsc Psychobiologie}
% {\customcvproject{Foundations of Neural and Cognitive Modelling}{Oct-Dec 2015}
% {Msc Artificial Intelligence, Master of Logic, Msc Brain and Cognitive Science}
% {\customcvproject{Unsupervised Language Learning}{Feb-Apr 2014, Feb-Apr 2015}
% {Msc Artificial Intelligence, Master of Logic}
% {\customcvproject{Automata and Formal Languages}{Apr-Jun 2013, Apr-Jun 2012}
% {Bsc Artificial Intelligence}
% {\customcvproject{Biomechanics}{Feb-Jun 2013}
% {Bsc bewegingswetenschappen (human motion sciences)}
% {\customcvproject{Logica}{Jan 2012, Jan 2013, Jan 2014}
% {Bsc Beta Gamma}


\section{AWARDS AND FELLOWSHIPS}

{\customcvproject{Honorary mention for best paper award, CoNLL2019}{November 2019}
{Analysing neural language models: contextual decomposition reveals default reasoning in number and gender assignment}

{\customcvproject{Research Internship}{January 2019}
{With Marco Baroni, at Facebook AI Research}

{\customcvproject{Best Paper Award, BlackboxNLP 2018}{August 2018}
{Under the hood: using diagnostic classifiers to investigate and improve how language models track agreement information}

{\customcvproject{Scholarship for Doctorate Studies}{June 2015}
{With Willem Zuidema, in the Language in Interaction Consortium}

{\customcvproject{Pre-PhD fellowship}{June 2015}
{With Rens Bod, within CREATE}


\section{SERVICES}
\subsection{Organisation}

{\customcventry{November 2020}{\href{https://blackboxnlp.github.io}{EMNLP workshop on analysing and interpreting neural networks}}{BlackboxNLP 2020 - Analyzing an interpreting Neural Networks}{Virtual}{}{}}

{\customcventry{August 2019}{\href{https://www.lorentzcenter.nl/lc/web/2019/1120/info.php3?wsid=1120}{Workshop at the Lorentz workshop on compositionality}}{Compositionality in Brains and Machines}{Leiden}
{\textbf{Role}: Lead organisor with Willem Zuidema and Marco Baroni. Tasks including writing workshop proposal, designing the workshop programme, inviting participants, being contact person for the Lorentz center and leading discussions during the workshop.}{}}

{\customcventry{August 2019}{\href{https://blackboxnlp.github.io/2019/}{ACL workshop on analysing and interpreting neural networks}}{BlackboxNLP 2019 - Analyzing an interpreting Neural Networks}{Florence}{\textbf{Role}: Co-organisation with Yonatan Belinkov, Grzegorz Chrupala and Tal Linzen. Tasks including sending out workshop calls, assigning reviewers, selecting papers, maintainance of workshop website and chairing sessions during the workshop.}{}}

{\customcventry{December 2017}{\href{https://smartcs.uva.nl/content/events/workshops/smart-conference-2017/12/workshop-4.html}{SMART workshop in honour of the scientific legacy of Remko Scha}}{Grammar, Computation and Cognition}{Amsterdam}{\textbf{Role}: Co-organisation with Willem Zuidema.}{}}

\subsection{Reviewing}
\begin{itemize}
    \item ICLR
    \item TACL
    \item Gecko
    \item CoNLL
    \item EMNLP
    \item BlackboxNLP 2019, BlackboxNLP 2020
    \item[]
\end{itemize}

\subsection{Area Chair}
\begin{itemize}
    \item EACL2020, interpretability track
    \item NAACL2021, interpretability track
    \item BlackboxNLP 2019, BlackboxNLP 2020
    \item[]
\end{itemize}

\subsection{Other}
\begin{itemize}
    \item PhD council ILLC, University of Amsterdam
    \item PhD council Faculty of Science, University of Amsterdam
\end{itemize}


\section{HIGHLIGHTED TALKS \& PANELS}

A full overview of the talks I have given can be found on my website at \textcolor{blue}{\url{https://dieuwkehupkes.nl/talks/}}.\\

\begin{itemize}
    \setlength\itemsep{5pt}
    \item \textit{October 10, 2023.} ChatGPT and other generative AI tools: risks and benefits, \textbf{Council of the European Union, Brussels} (pitch and panel discussion)
    \item \textit{April 11, 2023.} The (un?)importance of generalisation in NLP, \textbf{MSR, Montreal} (virtual talk)
    \item \textit{January 27, 2023.} GenBench: State-of-the-art generalisation research in NLP, \textbf{University of Cambridge} (virtual talk)
    \item \textit{June 29-30, 2022.} Are Neural Networks Compositional, and How Do We Even Know? \textbf{\href{https://compositionalintelligence.github.io/}{The Challenge of Compositionality for AI} (virtual talk \& panel)
    \item \textit{June 15, 2022.} Evaluating generalisation in natural language processing models}, \textbf{ODSC Europe, London} (London)
    \item \textit{April 14, 2022.} Evaluating generalisation in neural networks for NLP, \textbf{Stanford, Palo Alto}
    \item \textit{March 17, 2021.} Kunnen we kunstmatige intelligentie nog doorgronden? Studium Generale, \textbf{Studium Generale, Utrecht} (virtual talk)
    \item \textit{February 11, 2021.}, Compositionality decomposed: how do neural networks generalise? \textbf{Women@CL, University of Cambridge} (virtual talk)
    \item \textit{October 31, 2020.} Neural networks as explanatory models of language processing, \textbf{ILCC, University of Edinburgh} (virtual talk)
    \item \textit{September 17, 2020.} Neural networks as explanatory models, \textbf{AllenNLP, Seattle} (virtual talk)
    \item \textit{October 9, 2019.} Subject verb agreement in neural language models -- how, when and where? \textbf{Johns Hopkins University, Baltimore}
    \item \textit{September 3, 2019.} Guest speaker and panelist at the public event When fake looks all too real: the technology behind Deep Fake, \textbf{SPUI25, Amsterdam}.
    \item \textit{July 18, 2018}. Visualisation and ‘diagnostic classifiers’ reveal how recurrent and recursive neural networks process hierarchical structure. \textbf{IJCAI, Stockholm}.
    \item \textit{June 12, 2018}. Learning compositionally through attentive guidance. \textbf{University of Copenhagen.}
    \item \textit{May 9, 2017}. Processing hierarchical structure with RNNs.\textbf{ Dagstuhl}
    \item \textit{December 7, 2017}. The grammar of neural networks. \textbf{SMART workshop Grammars, Computation \& Cognition, Amsterdam.}
    \item \textit{December 15, 2017}. Hierarchical compositionality in recurrent neural networks. \textbf{Invited internal seminar at Rijksuniversiteit Groningen.}
    \item \textit{May 25, 2016}. POS-tagging of Historical Dutch. \textbf{LREC, Portoroz}.
    \item \textit{November 22, 2016}. How may neural networks process hierarchical structure? Insights from recursive and recurrent networks learning arithmetics. \textbf{Logic Tea, University of Amsterdam.}
    \item \textit{June 8, 2015} Using Parallel Data to improve Part-of-Speech tagging of 17th century Dutch. \textbf{DH Benelux, Antwerp.}
\end{itemize}


\section{PUBLICATIONS}

\begin{itemize}
\setlength\itemsep{5pt}
    \item Hupkes D., Rodriguez Luna D., Ponti E.M. and Bruni E. Internal and External Pressures on Language Emergence:
Least Effort, Object Constancy and Frequency. \textit{In prep}.
    \item Dagan, G., Hupkes, D. and Bruni E. Co-evolution of language and agent. \textit{In prep.}
    \item Hupkes D., Dankers V., Mul M. and Bruni E. \href{https://arxiv.org/pdf/1908.08351.pdf}{The compositionality of neural networks: integrating symbolism and connectionism}. \textit{Accepted for publication in JAIR.}
    
    \item Jumelet J., Zuidema W. and Hupkes D. (2019). \href{https://arxiv.org/pdf/1909.08975.pdf}{Analysing Neural Language Models: Contextual Decomposition Reveals Default Reasoning in Number and Gender Assignment}. \textit{CONLL 2019}.
    
    \item Baan J., Leible J., Nikolaus M., Rau D., Ulmer D., Baumgärtner T., Hupkes D. and Bruni E. (2019). \href{https://www.aclweb.org/anthology/W19-4814}{On the Realization of Compositionality in Neural Networks}. \textit{BlackboxNLP, ACL 2019}. 
    
    \item Ulmer D., Hupkes, D. and Bruni, E (2019). \href{https://www.aclweb.org/anthology/W19-4324}{Assessing incrementality in sequence-to-sequence models}. \textit{Repl4NLP, ACL 2019}. 
    
    \item Korrel K., Hupkes D., Dankers V., and Bruni E. (2019) \href{https://www.aclweb.org/anthology/W19-4801}{Transcoding compositionally: using attention to find more generalizable solutions}. \textit{BlackboxNLP, ACL 2019}.
    
    \item Lakretz Y., Kruszewski G., Desbordes T., Hupkes D., Dehaene S. and Baroni M. (2019) \href{https://www.aclweb.org/anthology/N19-1002}{The emergence of number and syntax units in LSTM language models}. \textit{NAACL 2019}.

    \item  Hupkes D., Singh A.K., Korrel K., Kruszewski G. and, Bruni E (2019). \href{https://arxiv.org/abs/1805.09657}{Learning compositionally through attentive guidance}. \textit{CICLing 2019}.
    
    \item Leonandya R., Bruni E., Hupkes D. and Kruszewski G (2019). \href{https://www.aclweb.org/anthology/W19-0419/}{The Fast and the Flexible: training neural networks to learn to follow instructions from small data.} \textit{IWCS}. 
    
    \item Hupkes D., Veldhoen S., and Zuidema W. (2018). \href{https://jair.org/index.php/jair/article/view/11196/26408}{Visualisation and ‘diagnostic classifiers’ reveal how recurrent and recursive neural networks process hierarchical structure}. \textit{Journal of Artificial Intelligence Research}.
    
    \item Giulianelli, M., Harding, J., Mohnert, F., Hupkes, D. and Zuidema, W. (2018). \href{https://aclweb.org/anthology/W18-5426}{Under the hood: using diagnostic classifiers to investigate and improve how language models track agreement information.} \textit{BlackboxNLP 2018, ACL}. \\ \textbf{Best paper award}.
    
    \item Zuidema W., Hupkes D., Wiggins G., Scharf C. and Rohrmeier M. (2018). Formal models of \href{https://arxiv.org/abs/1901.05180}{Structure Building in Music, Language and Animal Song}. In \textit{The Origins of Musicality}.
    
    \item Jumelet, J. and Hupkes, D. (2018). \href{https://aclweb.org/anthology/W18-5424}{Do language models understand anything? On the ability of LSTMs to understand negative polarity items}. \textit{BlackboxNLP 2018, ACL}.
    
    \item Hupkes, D., Bouwmeester, S. and Fernández, R. (2018). \href{https://aclweb.org/anthology/W18-5419}{Analysing the potential of seq2seq models for incremental interpretation in task-oriented dialogue}. \textit{BlackboxNLP 2018, ACL}.
    
    \item Hupkes D. and Zuidema W. (2017) \href{http://www.interpretable-ml.org/nips2017workshop/papers/12.pdf}{Diagnostic classification and symbolic guidance to understand and improve recurrent neural networks}. \textit{Interpreting, Explaining and Visualizing Deep Learning, NIPS2017}.
    
    \item Veldhoen S., Hupkes D. and Zuidema W. (2016). \href{http://dieuwkehupkes.nl/research/nips2016.pdf}{Diagnostic Classifiers: Revealing how Neural
Networks Process Hierarchical Shructure}. \textit{Cognitive Computation: Integrating Neural and
Symbolic Approaches, NIPS2016}.

    \item Hupkes D. and Bod R (2016) \href{https://www.aclweb.org/anthology/L16-1012}{POS-tagging of Historical Dutch}. \textit{LREC2016}.
\end{itemize}


\section{LANGUAGES}

\begin{tabular}{rlcrl}
\textsc{Dutch:} & Mothertongue && \textsc{French:} & Basic Knowledge\\
\textsc{English:} & Fluent && \textsc{Russian:} & Basic Knowledge\\
\textsc{Italian:} & Good command && \textsc{German:} & Basic Knowledge\\
\textsc{Spanish} & Basic command && \textsc{Frysian:} & Understanding\\
&\\
\end{tabular}



\section{ET CETERA}

I'm part of a competitive pole dance duo, we participated twice in the national championships, placing 1st and 2nd. In 2018, we also participated in the World Championship and arrived at place 10 of the world ranking. \url{https://instagram.com/duo_polenotti}.

% \section{REFERENCES}
% 
\begin{tabular}{ll}
\emph{Dr. Willem Zuidema} & \emph{Dr Marco Baroni} \\
Institute for Logic, Language and Computation & ICREA, Facebook AI Research\\
Universiteit van Amsterdam & \\
Room F2.45 & \\
Building F & \\
Science Park 107 & \\
1098 XG Amsterdam & \\
\href{mailto:w.h.zuidema@uva.nl}{w.h.zuidema@uva.nl}& \href{mailto:mbaroni@fb.com}{mbaroni@fb.com }\\\\
% \emph{Dr. K. Sima'an} & \emph{Prof. Dr. R. Bod} \\
% Institute for Logic, Language and Computation & Institute for Logic, Language and Computation\\
% Universiteit van Amsterdam & Universiteit van Amsterdam\\
% Room F2.06 & Room F2.04\\
% Building F & Building F\\
% Science Park 107 & Science Park 107\\
%1098 XG Amsterdam & 1098 XG Amsterdam\\
%\href{mailto:k.simaan@uva.nl}{k.simaan@uva.nl}& \href{mailto:rens.bod@gmail.com}{rens.bod@gmail.com}
\end{tabular}

% Publications from a BibTeX file without multibib
%  for numerical labels: \renewcommand{\bibliographyitemlabel}{\@biblabel{\arabic{enumiv}}}% CONSIDER MERGING WITH PREAMBLE PART
%  to redefine the heading string ("Publications"): \renewcommand{\refname}{Articles}
% \nocite{*}
% \bibliographystyle{plain}
% \bibliography{publications}                        % 'publications' is the name of a BibTeX file

% Publications from a BibTeX file using the multibib package
%\section{Publications}
%\nocitebook{book1,book2}
%\bibliographystylebook{plain}
%\bibliographybook{publications}                   % 'publications' is the name of a BibTeX file
%\nocitemisc{misc1,misc2,misc3}
%\bibliographystylemisc{plain}
%\bibliographymisc{publications}                   % 'publications' is the name of a BibTeX file

%-----       letter       ---------------------------------------------------------

\end{document}
