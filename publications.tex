\section{HIGHLIGHTED PUBLICATIONS}

A full overview of my publications can be found at my website (\textcolor{blue}{\url{https://dieuwkehupkes.nl/publications/}} or on my \textcolor{blue}{\href{https://scholar.google.com/citations?user=tAtSMTcAAAAJ&hl=en&oi=ao}{Google Scholar Page}}.\\

\begin{itemize}
\setlength\itemsep{5pt}
\item \textbf{D. Hupkes}, M. Giulianelli, V. Dankers, et. al (2023). \href{https://www.nature.com/articles/s42256-023-00729-y}{A taxonomy and review of generalization research in NLP}. Nature.
\item V. Dankers, E. Bruni, \textbf{D. Hupkes} \href{https://aclanthology.org/2022.acl-long.286/}{The paradox of the compositionality of natural language: a neural machine translation case study}. ACL.
\item V. Dankers, A. Langedijk, K. McCurdy, A. Williams, \textbf{D. Hupkes} (2021). \href{https://aclanthology.org/2021.conll-1.8/}{Generalising to German plural noun classes, from the perspective of a recurrent neural network}, CoNLL. \textbf{Best paper award}
\item \textbf{D. Hupkes}, V. Dankers, M. Mul, E. Bruni (2020). \href{https://www.jair.org/index.php/jair/article/view/11674/26576}{Compositionality decomposed:  how do neural networks generalise?}. JAIR.
\item \textbf{D. Hupkes}, S. Veldhoen, Zuidema, W. (2018). \href{https://jair.org/index.php/jair/article/view/11196/26408}{Visualisation and ‘diagnostic classifiers’ reveal how recurrent and recursive neural networks process hierarchical structure}. JAIR.
    \item Giulianelli, M., Harding, J., Mohnert, F., Hupkes, D. and Zuidema, W. (2018). \href{https://aclweb.org/anthology/W18-5426}{Under the hood: using diagnostic classifiers to investigate and improve how language models track agreement information.} \textit{BlackboxNLP 2018, ACL}. \textbf{Best paper award}.
\end{itemize}
